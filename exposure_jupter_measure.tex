\documentclass{article}
\usepackage{listings}
\usepackage{xcolor}
\usepackage{hyperref}

\definecolor{codegreen}{rgb}{0,0.6,0}
\definecolor{codegray}{rgb}{0.5,0.5,0.5}
\definecolor{codepurple}{rgb}{0.58,0,0.82}
\definecolor{backcolour}{rgb}{0.95,0.95,0.92}

\lstdefinestyle{mystyle}{
    backgroundcolor=\color{backcolour},   
    commentstyle=\color{codegreen},
    keywordstyle=\color{magenta},
    numberstyle=\tiny\color{codegray},
    stringstyle=\color{codepurple},
    basicstyle=\ttfamily\footnotesize,
    breakatwhitespace=false,         
    breaklines=true,                 
    captionpos=b,                    
    keepspaces=true,                 
    numbers=left,                    
    numbersep=5pt,                  
    showspaces=false,                
    showstringspaces=false,
    showtabs=false,                  
    tabsize=2
}

\lstset{style=mystyle}

\title{JupterExposureMeasure Detailed Explanation}
\author{}
\date{}

\begin{document}

\maketitle

\section{Overview}

\texttt{JupterExposureMeasure} is a concrete implementation of the \texttt{ExposureMeasure} abstract base class. It provides specific logic for calculating exposure measures based on Jupiter data for various hazard types.

\section{Class Definition}

\begin{lstlisting}[language=Python]
class JupterExposureMeasure(ExposureMeasure):
    def __init__(self):
        self.exposure_bins = self.get_exposure_bins()

    # Other methods...
\end{lstlisting}

\section{Key Methods}

\subsection{\texttt{\_\_init\_\_(self)}}
Initializes the exposure bins by calling \texttt{get\_exposure\_bins()}.

\subsection{\texttt{get\_data\_requests(self, asset: Asset, *, scenario: str, year: int) -> Iterable[HazardDataRequest]}}
Generates data requests for each hazard type and indicator.

\begin{itemize}
    \item Parameters:
    \begin{itemize}
        \item \texttt{asset}: The asset for which to request data
        \item \texttt{scenario}: Climate scenario
        \item \texttt{year}: Year for which to request data
    \end{itemize}
    \item Returns: An iterable of \texttt{HazardDataRequest} objects
\end{itemize}

Notable features:
\begin{itemize}
    \item Uses a specific model for wind data
    \item Requests data for all hazard types defined in \texttt{exposure\_bins}
\end{itemize}

\subsection{\texttt{get\_exposures(self, asset: Asset, data\_responses: Iterable[HazardDataResponse]) -> Dict[type, Tuple[Category, float, str]]}}
Calculates exposures based on the received hazard data.

\begin{itemize}
    \item Parameters:
    \begin{itemize}
        \item \texttt{asset}: The asset for which to calculate exposure
        \item \texttt{data\_responses}: Iterable of hazard data responses
    \end{itemize}
    \item Returns: A dictionary mapping hazard types to exposure categories, values, and data paths
\end{itemize}

Key logic:
\begin{itemize}
    \item Handles both \texttt{HazardParameterDataResponse} and \texttt{HazardEventDataResponse}
    \item Uses \texttt{np.searchsorted} to efficiently categorize exposure levels
    \item Assigns \texttt{Category.NODATA} for NaN values
\end{itemize}

\subsection{\texttt{get\_exposure\_bins(self) -> Dict}}
Defines the exposure bins for various hazard types.

\begin{itemize}
    \item Returns: A dictionary mapping (hazard\_type, indicator\_id) to (lower\_bounds, categories)
\end{itemize}

Hazard types covered:
\begin{itemize}
    \item CombinedInundation
    \item ChronicHeat
    \item Wind
    \item Drought
    \item Hail
    \item Fire
\end{itemize}

\subsection{\texttt{bounds\_to\_lookup(self, bounds: Iterable[Bounds]) -> Tuple[np.ndarray, np.ndarray]}}
Converts \texttt{Bounds} objects to numpy arrays for efficient lookup.

\begin{itemize}
    \item Parameters:
    \begin{itemize}
        \item \texttt{bounds}: Iterable of \texttt{Bounds} objects
    \end{itemize}
    \item Returns: Tuple of (lower\_bounds, categories) as numpy arrays
\end{itemize}

\section{Key Features}

\begin{enumerate}
    \item \textbf{Hazard-Specific Categorization}: 
    Defines specific exposure categories for each hazard type, allowing for tailored risk assessment.

    \item \textbf{Efficient Data Structures}: 
    Uses numpy arrays for efficient categorization of exposure levels.

    \item \textbf{Flexibility}: 
    Can handle different types of hazard data responses (parameter and event data).

    \item \textbf{Standardized Output}: 
    Returns exposure results in a consistent format across all hazard types.
\end{enumerate}

\section{Usage Example}

\begin{lstlisting}[language=Python]
exposure_measure = JupterExposureMeasure()

# Generate data requests for an asset
data_requests = exposure_measure.get_data_requests(asset, scenario="RCP8.5", year=2050)

# Assuming we have hazard_data_responses from a hazard model
exposures = exposure_measure.get_exposures(asset, hazard_data_responses)

# exposures will be a dictionary like:
# {
#     Wind: (Category.MEDIUM, 95.5, "wind/jupiter/v1/max_1min_RCP8.5_2050"),
#     ChronicHeat: (Category.HIGH, 25.3, "chronic_heat/days_above_35c"),
#     # ... other hazards ...
# }
\end{lstlisting}

\section{Considerations and Potential Improvements}

\begin{enumerate}
    \item \textbf{Configurability}: The exposure bins are hardcoded. Consider making them configurable through external settings.

    \item \textbf{Extensibility}: Adding new hazard types requires modifying \texttt{get\_exposure\_bins()}. A more dynamic approach could be beneficial.

    \item \textbf{Error Handling}: Additional error checking could be implemented, especially for unexpected data formats.

    \item \textbf{Performance}: For large-scale applications, consider optimizing the data processing, possibly using vectorized operations.

    \item \textbf{Documentation}: Adding docstrings to methods would improve code readability and maintainability.
\end{enumerate}

\end{document}