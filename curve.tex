\documentclass{article}
\usepackage{listings}
\usepackage{hyperref}
\usepackage{amsmath}
\usepackage{amssymb}

\title{Curve Module Documentation}
\author{}
\date{}

\begin{document}

\maketitle

\section{Overview}

This module provides utilities for working with exceedance curves and probability distributions. It includes functions for manipulating and processing bin edges and probabilities, as well as a class \texttt{ExceedanceCurve} for representing and working with exceedance curves.

\section{Functions}

\subsection{\texttt{add\_x\_value\_to\_curve(x, curve\_x, curve\_y)}}

Adds an x value to a curve, interpolated from the existing curve.

\begin{itemize}
    \item Parameters:
    \begin{itemize}
        \item \texttt{x}: The x value to add
        \item \texttt{curve\_x}: The existing x values of the curve (sorted non-decreasing)
        \item \texttt{curve\_y}: The existing y values of the curve
    \end{itemize}
    \item Returns: Updated \texttt{curve\_x} and \texttt{curve\_y}
\end{itemize}

\subsection{\texttt{to\_exceedance\_curve(bin\_edges, probs)}}

Converts bin edges and probabilities to an exceedance curve.

\begin{itemize}
    \item Parameters:
    \begin{itemize}
        \item \texttt{bin\_edges}: The edges of the bins
        \item \texttt{probs}: The probabilities for each bin
    \end{itemize}
    \item Returns: An \texttt{ExceedanceCurve} object
\end{itemize}

\subsection{\texttt{process\_bin\_edges\_and\_probs(bin\_edges, probs, range\_fraction=0.05)}}

Processes bin edges and probabilities to handle zero-width bins.

\begin{itemize}
    \item Parameters:
    \begin{itemize}
        \item \texttt{bin\_edges}: The edges of the bins
        \item \texttt{probs}: The probabilities for each bin
        \item \texttt{range\_fraction}: Fraction of range to use for zero-width bins (default: 0.05)
    \end{itemize}
    \item Returns: Processed bin edges and probabilities
\end{itemize}

\subsection{\texttt{process\_bin\_edges\_for\_graph(bin\_edges, range\_fraction=0.05)}}

Processes bin edges for graph display, handling zero-width bins.

\begin{itemize}
    \item Parameters:
    \begin{itemize}
        \item \texttt{bin\_edges}: The edges of the bins
        \item \texttt{range\_fraction}: Fraction of range to use for zero-width bins (default: 0.05)
    \end{itemize}
    \item Returns: Processed bin edges
\end{itemize}

\subsection{\texttt{\_\_next\_non\_equal\_index(ndarray, i)}}

Helper function to find the next index in an array with a different value.

\begin{itemize}
    \item Parameters:
    \begin{itemize}
        \item \texttt{ndarray}: The input array
        \item \texttt{i}: The starting index
    \end{itemize}
    \item Returns: The next index with a different value
\end{itemize}

\section{Class: ExceedanceCurve}

Represents an exceedance curve, where each point comprises a value \texttt{v} and a probability \texttt{p}. The probability \texttt{p} represents the chance that the random variable is greater than or equal to \texttt{v}.

\subsection{Methods}

\subsubsection{\texttt{\_\_init\_\_(self, probs, values)}}

Initializes the ExceedanceCurve.

\begin{itemize}
    \item Parameters:
    \begin{itemize}
        \item \texttt{probs}: Exceedance probabilities (must be sorted and decreasing)
        \item \texttt{values}: Corresponding values (must be sorted and non-decreasing)
    \end{itemize}
\end{itemize}

\subsubsection{\texttt{add\_value\_point(self, value)}}

Adds a point to the curve with the specified value.

\begin{itemize}
    \item Parameters:
    \begin{itemize}
        \item \texttt{value}: The value to add
    \end{itemize}
    \item Returns: A new ExceedanceCurve with the added point
\end{itemize}

\subsubsection{\texttt{get\_value(self, prob)}}

Gets the value corresponding to a given probability.

\begin{itemize}
    \item Parameters:
    \begin{itemize}
        \item \texttt{prob}: The probability
    \end{itemize}
    \item Returns: The corresponding value
\end{itemize}

\subsubsection{\texttt{get\_probability\_bins(self, include\_last=False)}}

Converts from exceedance probability to bins of constant probability density.

\begin{itemize}
    \item Parameters:
    \begin{itemize}
        \item \texttt{include\_last}: Whether to include the last bin (default: False)
    \end{itemize}
    \item Returns: Tuple of value bins and probabilities
\end{itemize}

\subsubsection{\texttt{get\_samples(self, uniforms)}}

Returns values for given uniform random variables.

\begin{itemize}
    \item Parameters:
    \begin{itemize}
        \item \texttt{uniforms}: Array of uniform random variables
    \end{itemize}
    \item Returns: Array of corresponding values
\end{itemize}

\section{Usage Notes}

\begin{enumerate}
    \item The \texttt{ExceedanceCurve} class is central to this module and provides various methods for working with exceedance curves.

    \item The \texttt{add\_x\_value\_to\_curve} function is useful for aligning curves and bins, but care should be taken with multiple identical x values.

    \item The \texttt{process\_bin\_edges\_and\_probs} and \texttt{process\_bin\_edges\_for\_graph} functions are helpful for handling zero-width bins, which can be problematic in certain analyses or visualizations.

    \item When creating an \texttt{ExceedanceCurve}, ensure that the probabilities are sorted and decreasing, and the values are sorted and non-decreasing.

    \item The \texttt{get\_samples} method of \texttt{ExceedanceCurve} can be used for Monte Carlo simulations or other sampling-based analyses.

    \item This module relies heavily on numpy for efficient array operations. Users should be familiar with numpy array handling for optimal use.
\end{enumerate}

\end{document}