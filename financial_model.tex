\documentclass{article}
\usepackage{listings}
\usepackage{xcolor}
\usepackage{hyperref}

\definecolor{codegreen}{rgb}{0,0.6,0}
\definecolor{codegray}{rgb}{0.5,0.5,0.5}
\definecolor{codepurple}{rgb}{0.58,0,0.82}
\definecolor{backcolour}{rgb}{0.95,0.95,0.92}

\lstdefinestyle{mystyle}{
    backgroundcolor=\color{backcolour},   
    commentstyle=\color{codegreen},
    keywordstyle=\color{magenta},
    numberstyle=\tiny\color{codegray},
    stringstyle=\color{codepurple},
    basicstyle=\ttfamily\footnotesize,
    breakatwhitespace=false,         
    breaklines=true,                 
    captionpos=b,                    
    keepspaces=true,                 
    numbers=left,                    
    numbersep=5pt,                  
    showspaces=false,                
    showstringspaces=false,
    showtabs=false,                  
    tabsize=2
}

\lstset{style=mystyle}

\title{Financial Model Module Documentation}
\author{}
\date{}

\begin{document}

\maketitle

\section{Overview}

This module defines the structure and implementations for financial modeling in a risk assessment context. It includes abstract base classes for financial data providers and financial models, as well as concrete implementations.

\section{Classes}

\subsection{FinancialDataProvider (ABC)}

An abstract base class that defines the interface for providing financial data about assets.

\subsubsection{Methods}

\begin{itemize}
    \item \texttt{get\_asset\_value(self, asset: Asset, currency: str) -> float}
    \begin{itemize}
        \item Returns the current value of the asset in the specified currency.
        \item Parameters:
        \begin{itemize}
            \item \texttt{asset}: The asset to value.
            \item \texttt{currency}: The currency in which to express the value.
        \end{itemize}
    \end{itemize}

    \item \texttt{get\_asset\_aggregate\_cashflows(self, asset: Asset, start: datetime, end: datetime, currency: str) -> float}
    \begin{itemize}
        \item Returns the expected sum of cashflows generated by the asset between the start and end dates, in the specified currency.
        \item Parameters:
        \begin{itemize}
            \item \texttt{asset}: The asset to calculate cashflows for.
            \item \texttt{start}: The start date for the cashflow calculation.
            \item \texttt{end}: The end date for the cashflow calculation.
            \item \texttt{currency}: The currency in which to express the cashflows.
        \end{itemize}
    \end{itemize}
\end{itemize}

\subsection{FinancialModelBase (ABC)}

An abstract base class that defines the interface for financial models used in risk assessment.

\subsubsection{Methods}

\begin{itemize}
    \item \texttt{damage\_to\_loss(self, asset: Asset, impact: np.ndarray, currency: str)}
    \begin{itemize}
        \item Converts the fractional damage of the specified asset to a financial loss.
        \item Parameters:
        \begin{itemize}
            \item \texttt{asset}: The affected asset.
            \item \texttt{impact}: An array of fractional damage values.
            \item \texttt{currency}: The currency in which to express the loss.
        \end{itemize}
    \end{itemize}

    \item \texttt{disruption\_to\_loss(self, asset: Asset, impact: np.ndarray, year: int, currency: str)}
    \begin{itemize}
        \item Converts the fractional annual disruption of the specified asset to a financial loss.
        \item Parameters:
        \begin{itemize}
            \item \texttt{asset}: The affected asset.
            \item \texttt{impact}: An array of fractional disruption values.
            \item \texttt{year}: The year for which to calculate the loss.
            \item \texttt{currency}: The currency in which to express the loss.
        \end{itemize}
    \end{itemize}
\end{itemize}

\subsection{FinancialModel}

A concrete implementation of \texttt{FinancialModelBase} that uses a \texttt{FinancialDataProvider} as its source of information.

\subsubsection{Attributes}

\begin{itemize}
    \item \texttt{data\_provider}: An instance of \texttt{FinancialDataProvider}
\end{itemize}

\subsubsection{Methods}

\begin{itemize}
    \item \texttt{\_\_init\_\_(self, data\_provider: FinancialDataProvider)}
    \begin{itemize}
        \item Initializes the FinancialModel with a data provider.
    \end{itemize}

    \item \texttt{damage\_to\_loss(self, asset: Asset, impact: np.ndarray, currency: str)}
    \begin{itemize}
        \item Implements the abstract method from \texttt{FinancialModelBase}.
        \item Calculates loss by multiplying the asset value by the impact.
    \end{itemize}

    \item \texttt{disruption\_to\_loss(self, asset: Asset, impact: np.ndarray, year: int, currency: str)}
    \begin{itemize}
        \item Implements the abstract method from \texttt{FinancialModelBase}.
        \item Calculates loss by multiplying the asset's annual cashflows by the impact.
    \end{itemize}
\end{itemize}

\subsection{CompositeFinancialModel}

A concrete implementation of \texttt{FinancialModelBase} that allows for different financial models to be used based on asset type.

\subsubsection{Attributes}

\begin{itemize}
    \item \texttt{financial\_models}: A dictionary mapping asset types to \texttt{FinancialModelBase} instances.
\end{itemize}

\subsubsection{Methods}

\begin{itemize}
    \item \texttt{\_\_init\_\_(self, financial\_models: Dict[type, FinancialModelBase])}
    \begin{itemize}
        \item Initializes the CompositeFinancialModel with a dictionary of financial models.
    \end{itemize}

    \item \texttt{damage\_to\_loss(self, asset: Asset, impact: np.ndarray, currency: str)}
    \begin{itemize}
        \item Delegates the calculation to the appropriate financial model based on the asset type.
    \end{itemize}

    \item \texttt{disruption\_to\_loss(self, asset: Asset, impact: np.ndarray, year: int, currency: str)}
    \begin{itemize}
        \item Delegates the calculation to the appropriate financial model based on the asset type.
    \end{itemize}
\end{itemize}

\section{Usage}

\begin{enumerate}
    \item Implement a concrete \texttt{FinancialDataProvider} to provide actual financial data for assets.
    \item Create instances of \texttt{FinancialModel} or \texttt{CompositeFinancialModel} as needed.
    \item Use these models in risk assessment calculations to convert physical impacts to financial losses.
\end{enumerate}

\section{Example}

\begin{lstlisting}[language=Python]
class MyDataProvider(FinancialDataProvider):
    def get_asset_value(self, asset, currency):
        # Implementation
        pass

    def get_asset_aggregate_cashflows(self, asset, start, end, currency):
        # Implementation
        pass

data_provider = MyDataProvider()
financial_model = FinancialModel(data_provider)

# Use in risk assessment
asset = Asset(...)
impact = np.array([0.1, 0.2, 0.3])  # Example impact values
loss = financial_model.damage_to_loss(asset, impact, "USD")
\end{lstlisting}

\section{Notes}

\begin{itemize}
    \item This module provides a flexible structure for financial modeling in risk assessment.
    \item The \texttt{CompositeFinancialModel} allows for tailored financial models for different asset types.
    \item Implementations should handle currency conversions if necessary.
    \item Error handling and input validation should be considered in concrete implementations.
\end{itemize}

\end{document}