\documentclass{article}
\usepackage{listings}
\usepackage{hyperref}
\usepackage{enumitem}

\title{Calculation Module Documentation}
\author{}
\date{}

\begin{document}

\maketitle

\section{Overview}

This module serves as a configuration hub for the physical risk assessment system. It defines default hazard models, vulnerability models, and risk measure calculators for various asset types. The module also includes a factory class for creating risk measure calculators based on specific use cases.

\section{Imports}

The module imports various components from the \texttt{physrisk} package, including hazard models, vulnerability models, and risk calculators. It also imports asset types from a local \texttt{.assets} module.

\section{Functions}

\subsection{\texttt{get\_default\_hazard\_model() -> HazardModel}}

Returns the default hazard model, which is a \texttt{ZarrHazardModel} that retrieves hazard event data from Zarr storage.

\subsection{\texttt{get\_default\_vulnerability\_models() -> Dict[type, Sequence[VulnerabilityModelBase]]}}

Returns a dictionary mapping asset types to sequences of vulnerability models. This function defines the default vulnerability models for different asset types:

\begin{itemize}
    \item \texttt{Asset}: Generic asset models (placeholder for unknown assets)
    \item \texttt{PowerGeneratingAsset}: Inundation model
    \item \texttt{RealEstateAsset}: Models for coastal and riverine inundation, tropical cyclones, and cooling
    \item \texttt{IndustrialActivity}: Chronic heat model
    \item \texttt{ThermalPowerGeneratingAsset}: Models for various thermal power plant vulnerabilities
    \item \texttt{TestAsset}: Temperature model
\end{itemize}

\subsection{\texttt{get\_default\_risk\_measure\_calculators() -> Dict[Type[Asset], RiskMeasureCalculator]}}

Returns a dictionary mapping asset types to risk measure calculators. Currently, it only defines a calculator for \texttt{RealEstateAsset}.

\section{Classes}

\subsection{\texttt{DefaultMeasuresFactory(RiskMeasuresFactory)}}

A factory class for creating risk measure calculators based on specific use cases.

Methods:
\begin{itemize}
    \item \texttt{calculators(self, use\_case\_id: str) -> Dict[Type[Asset], RiskMeasureCalculator]}:
    \begin{itemize}
        \item If \texttt{use\_case\_id} is "generic", returns a dictionary with a \texttt{GenericScoreBasedRiskMeasures} calculator for the \texttt{Asset} type.
        \item Otherwise, returns the result of \texttt{get\_default\_risk\_measure\_calculators()}.
    \end{itemize}
\end{itemize}

\section{Key Components}

\begin{enumerate}
    \item \textbf{Hazard Model}: The default hazard model uses Zarr storage for retrieving hazard event data.

    \item \textbf{Vulnerability Models}: 
    \begin{itemize}
        \item For generic assets, placeholder models are used for various hazards (fire, chronic heat, hail, drought, precipitation).
        \item Specific models are defined for power generating assets, real estate assets, industrial activities, and thermal power generating assets.
        \item These models cover a range of hazards including inundation, tropical cyclones, chronic heat, and various thermal power plant-specific vulnerabilities.
    \end{itemize}

    \item \textbf{Risk Measure Calculators}: 
    \begin{itemize}
        \item The default setup includes a specific calculator for real estate assets (\texttt{RealEstateToyRiskMeasures}).
        \item The \texttt{DefaultMeasuresFactory} allows for dynamic selection of risk measure calculators based on the use case.
    \end{itemize}
\end{enumerate}

\section{Usage Notes}

\begin{enumerate}
    \item This module serves as a central configuration point for the physical risk assessment system. Users can modify these functions to customize the models and calculators used in their assessments.

    \item The \texttt{get\_default\_vulnerability\_models()} function allows for easy extension to support new asset types or vulnerability models.

    \item The \texttt{DefaultMeasuresFactory} class provides a flexible way to select risk measure calculators based on different use cases. Users can extend this class to support additional use cases.

    \item When adding new asset types or models, ensure they are imported correctly and added to the appropriate dictionaries in this module.

    \item The use of placeholder vulnerability models (e.g., \texttt{PlaceholderVulnerabilityModel}) suggests that some parts of the system may still be under development or pending more specific implementations.
\end{enumerate}

\end{document}