\documentclass{article}
\usepackage{listings}
\usepackage{hyperref}
\usepackage{xcolor}

\definecolor{codegreen}{rgb}{0,0.6,0}
\definecolor{codegray}{rgb}{0.5,0.5,0.5}
\definecolor{codepurple}{rgb}{0.58,0,0.82}
\definecolor{backcolour}{rgb}{0.95,0.95,0.92}

\lstdefinestyle{mystyle}{
    backgroundcolor=\color{backcolour},   
    commentstyle=\color{codegreen},
    keywordstyle=\color{magenta},
    numberstyle=\tiny\color{codegray},
    stringstyle=\color{codepurple},
    basicstyle=\ttfamily\footnotesize,
    breakatwhitespace=false,         
    breaklines=true,                 
    captionpos=b,                    
    keepspaces=true,                 
    numbers=left,                    
    numbersep=5pt,                  
    showspaces=false,                
    showstringspaces=false,
    showtabs=false,                  
    tabsize=2
}

\lstset{style=mystyle}

\title{Events Numba Overview}
\author{}
\date{}

\begin{document}

\maketitle

\section{What is Numba?}

Numba is an open-source JIT compiler that translates a subset of Python and NumPy code into fast machine code. It's particularly useful for numerical and array-oriented computing.

\section{Key Features}

\begin{enumerate}
    \item \textbf{JIT Compilation}: Numba compiles Python functions to optimized machine code at runtime.

    \item \textbf{NumPy Integration}: Works seamlessly with NumPy arrays and functions.

    \item \textbf{GPU Acceleration}: Can target NVIDIA CUDA GPUs for parallel computing.

    \item \textbf{Automatic Optimization}: Applies various optimizations without requiring changes to your Python code.
\end{enumerate}

\section{How Numba is Used in the \texttt{events.py} File}

In the \texttt{events.py} file, Numba is used in several ways:

\begin{enumerate}
    \item \textbf{@njit Decorator}: 
    \begin{lstlisting}[language=Python]
@njit(cache=True)
def sample_from_cumulative_probs(values, cum_probs, uniforms):
    \end{lstlisting}
    This decorator compiles the function to machine code. The \texttt{cache=True} argument allows Numba to cache the compiled function for faster subsequent calls.

    \item \textbf{@jitclass Decorator}:
    \begin{lstlisting}[language=Python]
@jitclass(spec)
class CumulativeProb(object):
    \end{lstlisting}
    This decorator is used to compile a Python class, allowing for fast operations on its methods and attributes.

    \item \textbf{Performance Benefits}: By using Numba, computationally intensive functions like \texttt{sample\_from\_cumulative\_probs} and \texttt{event\_samples\_numba} can run much faster than standard Python code, especially when dealing with large arrays.
\end{enumerate}

\section{Advantages of Using Numba}

\begin{enumerate}
    \item \textbf{Speed}: Can significantly speed up numerical computations.
    \item \textbf{Ease of Use}: Requires minimal changes to existing Python code.
    \item \textbf{NumPy Compatibility}: Works well with existing NumPy code.
    \item \textbf{Selective Optimization}: Can be applied to specific functions that need performance boosts.
\end{enumerate}

\section{Considerations}

\begin{enumerate}
    \item \textbf{First-Run Overhead}: There's a compilation overhead on the first run of a Numba-optimized function.
    \item \textbf{Limited Python Subset}: Not all Python features are supported by Numba.
    \item \textbf{Debugging}: Can make debugging more challenging as the code is compiled.
\end{enumerate}

\end{document}