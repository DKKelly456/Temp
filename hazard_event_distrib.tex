\documentclass{article}
\usepackage{listings}
\usepackage{xcolor}
\usepackage{hyperref}

\definecolor{codegreen}{rgb}{0,0.6,0}
\definecolor{codegray}{rgb}{0.5,0.5,0.5}
\definecolor{codepurple}{rgb}{0.58,0,0.82}
\definecolor{backcolour}{rgb}{0.95,0.95,0.92}

\lstdefinestyle{mystyle}{
    backgroundcolor=\color{backcolour},   
    commentstyle=\color{codegreen},
    keywordstyle=\color{magenta},
    numberstyle=\tiny\color{codegray},
    stringstyle=\color{codepurple},
    basicstyle=\ttfamily\footnotesize,
    breakatwhitespace=false,         
    breaklines=true,                 
    captionpos=b,                    
    keepspaces=true,                 
    numbers=left,                    
    numbersep=5pt,                  
    showspaces=false,                
    showstringspaces=false,
    showtabs=false,                  
    tabsize=2
}

\lstset{style=mystyle}

\title{HazardEventDistrib Class Documentation}
\author{}
\date{}

\begin{document}

\maketitle

\section{Overview}

The \texttt{HazardEventDistrib} class represents the intensity distribution of a hazard event (such as inundation depth or wind speed) specific to the location of an asset. It provides methods to manage and analyze the probability distribution of hazard intensities.

\section{Class Definition}

\begin{lstlisting}[language=Python]
class HazardEventDistrib:
    __slots__ = ["__event_type", "__intensity_bins", "__prob", "__path", "__exceedance"]
\end{lstlisting}

The use of \texttt{\_\_slots\_\_} optimizes memory usage and provides faster attribute access.

\section{Constructor}

\begin{lstlisting}[language=Python]
def __init__(
    self,
    event_type: type,
    intensity_bins: Union[List[float], np.ndarray],
    prob: Union[List[float], np.ndarray],
    path: List[str]
):
\end{lstlisting}

\subsection{Parameters:}

\begin{itemize}
    \item \texttt{event\_type} (type): The type of hazard event.
    \item \texttt{intensity\_bins} (Union[List[float], np.ndarray]): Non-decreasing intensity bin edges.
    \item \texttt{prob} (Union[List[float], np.ndarray]): Annual probability of occurrence for each intensity bin.
    \item \texttt{path} (List[str]): Path to the hazard indicator data source.
\end{itemize}

\subsection{Behavior:}

\begin{itemize}
    \item Initializes the class attributes with the provided values.
    \item Converts \texttt{intensity\_bins} and \texttt{prob} to numpy arrays for efficient computation.
\end{itemize}

\section{Properties}

\subsection{\texttt{intensity\_bin\_edges}}

\begin{lstlisting}[language=Python]
@property
def intensity_bin_edges(self) -> np.ndarray:
\end{lstlisting}

Returns the array of intensity bin edges.

\subsection{\texttt{prob}}

\begin{lstlisting}[language=Python]
@property
def prob(self) -> np.ndarray:
\end{lstlisting}

Returns the array of probabilities for each intensity bin.

\subsection{\texttt{path}}

\begin{lstlisting}[language=Python]
@property
def path(self) -> List[str]:
\end{lstlisting}

Returns the path to the hazard indicator data source.

\section{Methods}

\subsection{\texttt{intensity\_bins()}}

\begin{lstlisting}[language=Python]
def intensity_bins(self):
\end{lstlisting}

Returns a generator of tuples representing the lower and upper bounds of each intensity bin.

\subsection{\texttt{to\_exceedance\_curve()}}

\begin{lstlisting}[language=Python]
def to_exceedance_curve(self):
\end{lstlisting}

Converts the probability distribution to an exceedance curve using the \texttt{curve.to\_exceedance\_curve()} function.

\section{Usage Example}

\begin{lstlisting}[language=Python]
# Create a HazardEventDistrib instance
hazard_dist = HazardEventDistrib(
    event_type=WindEvent,
    intensity_bins=[0, 10, 20, 30, 40],
    prob=[0.5, 0.3, 0.15, 0.05],
    path=["wind_data", "location_123"]
)

# Access properties
print(hazard_dist.intensity_bin_edges)  # [0, 10, 20, 30, 40]
print(hazard_dist.prob)  # [0.5, 0.3, 0.15, 0.05]

# Iterate over intensity bins
for lower, upper in hazard_dist.intensity_bins():
    print(f"Bin: {lower} to {upper}")

# Convert to exceedance curve
exceedance_curve = hazard_dist.to_exceedance_curve()
\end{lstlisting}

\section{Notes}

\begin{enumerate}
    \item The class uses double underscore name mangling for its attributes, indicating that they are intended to be private.

    \item The \texttt{intensity\_bins} method returns a generator, which is memory-efficient for large datasets.

    \item The \texttt{to\_exceedance\_curve()} method relies on an external \texttt{curve} module, which should be imported and available.

    \item The class is designed to work with numpy arrays for efficient numerical operations.

    \item The use of \texttt{Union[List[float], np.ndarray]} in the constructor allows for flexibility in input types.
\end{enumerate}

\section{Potential Improvements}

\begin{enumerate}
    \item Add input validation in the constructor to ensure \texttt{intensity\_bins} and \texttt{prob} have compatible shapes.

    \item Implement additional methods for statistical analysis of the hazard distribution.

    \item Add documentation strings (docstrings) to methods for better inline documentation.

    \item Consider adding a method to visualize the probability distribution or exceedance curve.

    \item Implement serialization methods if persistence of these objects is required.
\end{enumerate}

\end{document}