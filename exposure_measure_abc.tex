\documentclass{article}
\usepackage{listings}
\usepackage{xcolor}
\usepackage{hyperref}

\definecolor{codegreen}{rgb}{0,0.6,0}
\definecolor{codegray}{rgb}{0.5,0.5,0.5}
\definecolor{codepurple}{rgb}{0.58,0,0.82}
\definecolor{backcolour}{rgb}{0.95,0.95,0.92}

\lstdefinestyle{mystyle}{
    backgroundcolor=\color{backcolour},   
    commentstyle=\color{codegreen},
    keywordstyle=\color{magenta},
    numberstyle=\tiny\color{codegray},
    stringstyle=\color{codepurple},
    basicstyle=\ttfamily\footnotesize,
    breakatwhitespace=false,         
    breaklines=true,                 
    captionpos=b,                    
    keepspaces=true,                 
    numbers=left,                    
    numbersep=5pt,                  
    showspaces=false,                
    showstringspaces=false,
    showtabs=false,                  
    tabsize=2
}

\lstset{style=mystyle}

\title{ExposureMeasure (ABC) Detailed Explanation}
\author{}
\date{}

\begin{document}

\maketitle

\section{Overview}

\texttt{ExposureMeasure} is an abstract base class (ABC) that defines the interface for exposure measure implementations. It serves as a blueprint for creating different types of exposure measures in the system.

\section{Class Definition}

\begin{lstlisting}[language=Python]
class ExposureMeasure(DataRequester):
    @abstractmethod
    def get_exposures(
        self, asset: Asset, data_responses: Iterable[HazardDataResponse]
    ) -> Dict[type, Tuple[Category, float, str]]: ...
\end{lstlisting}

\section{Key Aspects}

\begin{enumerate}
    \item \textbf{Inheritance from DataRequester}: 
    \texttt{ExposureMeasure} inherits from \texttt{DataRequester}, which likely defines methods for requesting data. This suggests that exposure measures are responsible for specifying what data they need.

    \item \textbf{Abstract Method}:
    The class defines one abstract method, \texttt{get\_exposures}, which must be implemented by any concrete subclass.

    \item \textbf{Method Signature}:
    \begin{itemize}
        \item \texttt{asset: Asset}: The asset for which exposure is being calculated.
        \item \texttt{data\_responses: Iterable[HazardDataResponse]}: Responses from hazard data requests.
        \item Returns \texttt{Dict[type, Tuple[Category, float, str]]}: A mapping of hazard types to exposure results.
    \end{itemize}

    \item \textbf{Return Value Structure}:
    \begin{itemize}
        \item \texttt{type}: Likely represents the hazard type (e.g., Flood, Fire).
        \item \texttt{Category}: An enum representing the exposure category (e.g., LOW, MEDIUM, HIGH).
        \item \texttt{float}: The numerical value of the exposure.
        \item \texttt{str}: Probably a path or identifier for the data source.
    \end{itemize}
\end{enumerate}

\section{Purpose and Design Philosophy}

\begin{enumerate}
    \item \textbf{Separation of Concerns}: 
    By defining an abstract base class, the module separates the interface of exposure measures from their implementations. This allows for different exposure calculation methodologies without changing the overall system structure.

    \item \textbf{Polymorphism}: 
    The abstract class enables polymorphic use of different exposure measures. Functions like \texttt{calculate\_exposures} can work with any concrete implementation of \texttt{ExposureMeasure}.

    \item \textbf{Extensibility}: 
    New exposure measures can be easily added by creating new classes that inherit from \texttt{ExposureMeasure} and implement the \texttt{get\_exposures} method.

    \item \textbf{Standardization}: 
    The ABC ensures that all exposure measures have a consistent interface, making the system more maintainable and easier to understand.
\end{enumerate}

\section{Usage and Implementation}

To create a new exposure measure:

\begin{enumerate}
    \item Create a new class that inherits from \texttt{ExposureMeasure}.
    \item Implement the \texttt{get\_exposures} method, ensuring it adheres to the defined signature.
    \item Optionally, implement any methods from \texttt{DataRequester} if needed.
\end{enumerate}

Example skeleton:

\begin{lstlisting}[language=Python]
class NewExposureMeasure(ExposureMeasure):
    def get_exposures(self, asset: Asset, data_responses: Iterable[HazardDataResponse]) -> Dict[type, Tuple[Category, float, str]]:
        # Implementation here
        pass

    # Optionally implement DataRequester methods
    def get_data_requests(self, asset: Asset, *, scenario: str, year: int) -> Iterable[HazardDataRequest]:
        # Implementation here
        pass
\end{lstlisting}

\section{Considerations for Implementers}

\begin{enumerate}
    \item \textbf{Data Handling}: Implementations need to handle various types of \texttt{HazardDataResponse} objects correctly.
    \item \textbf{Categorization Logic}: Define clear logic for categorizing exposure levels based on the data.
    \item \textbf{Error Handling}: Consider how to handle missing or invalid data in the responses.
    \item \textbf{Performance}: For large-scale applications, consider the efficiency of the implementation, especially if processing many assets.
\end{enumerate}

\end{document}