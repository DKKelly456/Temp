\documentclass{article}
\usepackage{listings}
\usepackage{xcolor}
\usepackage{hyperref}

\definecolor{codegreen}{rgb}{0,0.6,0}
\definecolor{codegray}{rgb}{0.5,0.5,0.5}
\definecolor{codepurple}{rgb}{0.58,0,0.82}
\definecolor{backcolour}{rgb}{0.95,0.95,0.92}

\lstdefinestyle{mystyle}{
    backgroundcolor=\color{backcolour},   
    commentstyle=\color{codegreen},
    keywordstyle=\color{magenta},
    numberstyle=\tiny\color{codegray},
    stringstyle=\color{codepurple},
    basicstyle=\ttfamily\footnotesize,
    breakatwhitespace=false,         
    breaklines=true,                 
    captionpos=b,                    
    keepspaces=true,                 
    numbers=left,                    
    numbersep=5pt,                  
    showspaces=false,                
    showstringspaces=false,
    showtabs=false,                  
    tabsize=2
}

\lstset{style=mystyle}

\title{CompositeFinancialModel Detailed Explanation}
\author{}
\date{}

\begin{document}

\maketitle

\section{Class Definition}

\begin{lstlisting}[language=Python]
class CompositeFinancialModel(FinancialModelBase):
    def __init__(self, financial_models: Dict[type, FinancialModelBase]):
        self.financial_models = financial_models

    def damage_to_loss(self, asset: Asset, impact: np.ndarray, currency: str):
        return self.financial_models[type(asset)].damage_to_loss(
            asset, impact, currency
        )

    def disruption_to_loss(
        self, asset: Asset, impact: np.ndarray, year: int, currency: str
    ):
        return self.financial_models[type(asset)].disruption_to_loss(
            asset, impact, year, currency
        )
\end{lstlisting}

\section{Purpose}

The \texttt{CompositeFinancialModel} is designed to provide a flexible way to apply different financial models to different types of assets within a single, unified interface. This is particularly useful in scenarios where various asset classes require distinct financial modeling approaches.

\section{Key Features}

\begin{enumerate}
    \item \textbf{Type-based Model Selection}: Uses the type of the asset to determine which specific financial model to apply.

    \item \textbf{Polymorphic Behavior}: Maintains the \texttt{FinancialModelBase} interface while allowing for diverse underlying implementations.

    \item \textbf{Extensibility}: Easily accommodates new asset types and corresponding financial models without modifying existing code.

    \item \textbf{Encapsulation}: Hides the complexity of multiple models from the user of the \texttt{CompositeFinancialModel}.
\end{enumerate}

\section{Detailed Breakdown}

\subsection{Constructor}

\begin{lstlisting}[language=Python]
def __init__(self, financial_models: Dict[type, FinancialModelBase]):
    self.financial_models = financial_models
\end{lstlisting}

\begin{itemize}
    \item \textbf{Parameter}: \texttt{financial\_models} is a dictionary where:
    \begin{itemize}
        \item Keys are Python types (presumably subclasses of \texttt{Asset})
        \item Values are instances of classes derived from \texttt{FinancialModelBase}
    \end{itemize}
    \item This structure allows for a flexible mapping of asset types to specific financial models.
\end{itemize}

\subsection{Method: damage\_to\_loss}

\begin{lstlisting}[language=Python]
def damage_to_loss(self, asset: Asset, impact: np.ndarray, currency: str):
    return self.financial_models[type(asset)].damage_to_loss(
        asset, impact, currency
    )
\end{lstlisting}

\begin{itemize}
    \item Uses \texttt{type(asset)} to select the appropriate financial model from the dictionary.
    \item Delegates the actual calculation to the selected model's \texttt{damage\_to\_loss} method.
    \item Maintains the same interface as \texttt{FinancialModelBase}, ensuring compatibility.
\end{itemize}

\subsection{Method: disruption\_to\_loss}

\begin{lstlisting}[language=Python]
def disruption_to_loss(
    self, asset: Asset, impact: np.ndarray, year: int, currency: str
):
    return self.financial_models[type(asset)].disruption_to_loss(
        asset, impact, year, currency
    )
\end{lstlisting}

\begin{itemize}
    \item Similar to \texttt{damage\_to\_loss}, but for disruption calculations.
    \item Again, delegates to the type-specific model's method.
\end{itemize}

\section{Usage Example}

\begin{lstlisting}[language=Python]
class RealEstateModel(FinancialModelBase):
    # Specific implementation for real estate assets

class InfrastructureModel(FinancialModelBase):
    # Specific implementation for infrastructure assets

composite_model = CompositeFinancialModel({
    RealEstateAsset: RealEstateModel(),
    InfrastructureAsset: InfrastructureModel()
})

# Usage
real_estate = RealEstateAsset(...)
infrastructure = InfrastructureAsset(...)

loss_re = composite_model.damage_to_loss(real_estate, [0.1, 0.2], "USD")
loss_infra = composite_model.damage_to_loss(infrastructure, [0.05, 0.15], "USD")
\end{lstlisting}

\section{Advantages}

\begin{enumerate}
    \item \textbf{Modularity}: Each asset type can have its own specialized financial model.
    \item \textbf{Single Interface}: Users interact with a single \texttt{CompositeFinancialModel}, simplifying the API.
    \item \textbf{Easy Maintenance}: New asset types and models can be added without changing existing code.
    \item \textbf{Separation of Concerns}: Each individual financial model can focus on its specific asset type.
\end{enumerate}

\section{Considerations and Potential Improvements}

\begin{enumerate}
    \item \textbf{Error Handling}: Add checks for missing asset types in the dictionary.
    \item \textbf{Default Model}: Consider providing a default model for unrecognized asset types.
    \item \textbf{Dynamic Registration}: Implement methods to add or remove models at runtime.
    \item \textbf{Validation}: Add checks to ensure all provided models adhere to the \texttt{FinancialModelBase} interface.
    \item \textbf{Documentation}: Include type hints and docstrings for better IDE support and user guidance.
\end{enumerate}

\end{document}