\documentclass{article}
\usepackage{listings}
\usepackage{xcolor}
\usepackage{hyperref}

\definecolor{codegreen}{rgb}{0,0.6,0}
\definecolor{codegray}{rgb}{0.5,0.5,0.5}
\definecolor{codepurple}{rgb}{0.58,0,0.82}
\definecolor{backcolour}{rgb}{0.95,0.95,0.92}

\lstdefinestyle{mystyle}{
    backgroundcolor=\color{backcolour},   
    commentstyle=\color{codegreen},
    keywordstyle=\color{magenta},
    numberstyle=\tiny\color{codegray},
    stringstyle=\color{codepurple},
    basicstyle=\ttfamily\footnotesize,
    breakatwhitespace=false,         
    breaklines=true,                 
    captionpos=b,                    
    keepspaces=true,                 
    numbers=left,                    
    numbersep=5pt,                  
    showspaces=false,                
    showstringspaces=false,
    showtabs=false,                  
    tabsize=2
}

\lstset{style=mystyle}

\title{calculate\_exposures Function Detailed Explanation}
\author{}
\date{}

\begin{document}

\maketitle

\section{Function Signature}

\begin{lstlisting}[language=Python]
def calculate_exposures(
    assets: List[Asset],
    hazard_model: HazardModel,
    exposure_measure: ExposureMeasure,
    scenario: str,
    year: int,
) -> Dict[Asset, AssetExposureResult]:
\end{lstlisting}

\section{Purpose}

The \texttt{calculate\_exposures} function is the main entry point for calculating exposure results for a list of assets. It orchestrates the process of requesting hazard data, applying the exposure measure, and collecting the results.

\section{Parameters}

\begin{enumerate}
    \item \texttt{assets: List[Asset]}: A list of assets for which to calculate exposures.
    \item \texttt{hazard\_model: HazardModel}: The model used to provide hazard data.
    \item \texttt{exposure\_measure: ExposureMeasure}: The measure used to calculate exposures (e.g., JupterExposureMeasure).
    \item \texttt{scenario: str}: The climate scenario to use (e.g., "RCP8.5").
    \item \texttt{year: int}: The year for which to calculate exposures.
\end{enumerate}

\section{Return Value}

\texttt{Dict[Asset, AssetExposureResult]}: A dictionary mapping each asset to its calculated exposure result.

\section{Key Steps}

\subsection{Data Request Consolidation}
\begin{lstlisting}[language=Python]
requester_assets: Dict[DataRequester, List[Asset]] = {exposure_measure: assets}
asset_requests, responses = _request_consolidated(
    hazard_model, requester_assets, scenario, year
)
\end{lstlisting}
\begin{itemize}
    \item Consolidates data requests for all assets using the given exposure measure.
    \item Uses a helper function \texttt{\_request\_consolidated} to efficiently request data from the hazard model.
\end{itemize}

\subsection{Logging}
\begin{lstlisting}[language=Python]
logging.info(
    "Applying exposure measure {0} to {1} assets of type {2}".format(
        type(exposure_measure).__name__, len(assets), type(assets[0]).__name__
    )
)
\end{lstlisting}
\begin{itemize}
    \item Logs information about the exposure calculation process.
\end{itemize}

\subsection{Exposure Calculation Loop}
\begin{lstlisting}[language=Python]
for asset in assets:
    requests = asset_requests[(exposure_measure, asset)]
    hazard_data = [responses[req] for req in get_iterable(requests)]
    result[asset] = AssetExposureResult(
        hazard_categories=exposure_measure.get_exposures(asset, hazard_data)
    )
\end{lstlisting}
\begin{itemize}
    \item Iterates through each asset.
    \item Retrieves the specific hazard data responses for the asset.
    \item Applies the exposure measure to calculate exposures.
    \item Stores the result in an \texttt{AssetExposureResult} object.
\end{itemize}

\section{Key Features}

\begin{enumerate}
    \item \textbf{Batch Processing}: Calculates exposures for multiple assets in a single function call.

    \item \textbf{Flexible Hazard Model}: Uses a generic \texttt{HazardModel} interface, allowing for different implementations.

    \item \textbf{Customizable Exposure Measure}: Accepts any implementation of \texttt{ExposureMeasure}, enabling different exposure calculation methodologies.

    \item \textbf{Efficient Data Retrieval}: Consolidates data requests to minimize redundant calls to the hazard model.

    \item \textbf{Scenario and Year Specific}: Calculates exposures for a specific climate scenario and year.
\end{enumerate}

\section{Usage Example}

\begin{lstlisting}[language=Python]
assets = [Asset(...), Asset(...), ...]  # List of assets
hazard_model = SomeHazardModel(...)  # Your hazard model implementation
exposure_measure = JupterExposureMeasure()
scenario = "RCP8.5"
year = 2050

exposure_results = calculate_exposures(assets, hazard_model, exposure_measure, scenario, year)

for asset, result in exposure_results.items():
    print(f"Asset: {asset.id}")
    for hazard_type, (category, value, path) in result.hazard_categories.items():
        print(f"  {hazard_type.__name__}: {category.name} (value: {value}, source: {path})")
\end{lstlisting}

\section{Considerations and Potential Improvements}

\begin{enumerate}
    \item \textbf{Parallelization}: For large numbers of assets, consider implementing parallel processing.

    \item \textbf{Error Handling}: Add more robust error handling, especially for cases where hazard data might be missing.

    \item \textbf{Progress Reporting}: For long-running calculations, implement a progress reporting mechanism.

    \item \textbf{Caching}: Consider caching hazard data responses to improve performance for repeated calculations.

    \item \textbf{Flexibility}: Allow for calculating exposures for different scenarios or years in a single call.

    \item \textbf{Validation}: Add input validation to ensure all required data is present and in the correct format.
\end{enumerate}

\end{document}